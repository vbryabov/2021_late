\section{レポート課題4}
\subsection{課題1}
リアプノフ指数の $r$ 依存性を示したグラフを描け。但し、初期値をランダムに与え、グラフの横軸は $1 〜 4$ 、縦軸は $-3 〜 1$ までの範囲にすること。 $r$ の刻み幅は、各自適切な値を設定すること。\\


\subsection{課題2}
3周期の窓の領域でのリアプノフ指数の $r$ 依存性を示したグラフを描け。但し、初期値をランダムに与え、グラフの縦軸は $-1 〜 0.4$ までの範囲にすること。 $r$ の範囲および刻み幅は、各自適切な値を設定すること。


\subsection{課題3}
ロジスティック写像についてまとめ、これまでに出題された全て (4 回分 ) の結果について考察せよ。分量は A4 用紙 1 〜 4 枚程度を目安としてください。\\\\
 レポート課題1の考察は、ロジスティック写像は $r$ の値を変化させていくことで非周期性を満たすものと満たさないものがあると考察した。レポート課題1では初期値は $x_0 = 0.7$ に固定して $r$ の値によってどのような軌道になっていくかを比較した。それぞれの結果から、 $x_{n+1} = r(1 −x_n)x_n$ と $x_{n+1} = x_n$、 $x_0$ の位置関係と $r$ の値が収束、発散と関係していると考察した。\\
 レポート課題2の考察は、 $r = 3.86, r = 3.90$ のときは初期値 $x_0$ の小さい変化によって $x_{200}$ と $x_n (150 < n < 200)$ が大きく変化することが考察できる。レポート課題2では、初期値 $x_0$ の変化によって $x_{200}$ がどのようになっていくかを調べる問題になっていた。また、レポート課題2で比較した $r$ はレポート課題1と同様のものとなっている。レポート課題2の結果から非周期性を満たすかどうかは $r$ によって決まっていくことがより強く考察することができた。また、今回は初期値 $x_0$ を変化させていたので非周期性を満たすときの $x_{200}$ の挙動がどうなっていくか確認することができた。非周期性を満たすときの $x_{200}$ の値には規則性が見られることはなかった。これにより $r = 3.86, r = 3.90$ のときには初期値鋭敏性の要素が含まれていることが考察できた。初期値鋭敏性とは、初期状態での小さな差があると時間経過に応じて指数関数的にその差が増加するというカオスの条件のひとつである。\\
 レポート課題3は、添付している画像のランダムの値の他にもいくつかの値で課題1を実行してみたが、やはり $r = 1.50, r = 2.60, r = 3.20, r = 3.50$ のときはある値を反復し $r = 3.86, r = 3.90$ のときは値が定まることなく変化していた。これらの結果から $r = 3.86, r = 3.90$ のときはカオスの条件のひとつである非周期性が満たされていると考察することができる。また課題2の結果からこれらの結果から初期値 $x_0$ ではなく $r$ の値がカオスの条件と関係していると考察することができる。\\
 レポート課題4\\